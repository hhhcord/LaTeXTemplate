\documentclass[a4]{jarticle}  %文字サイズ・用紙サイズ・段組設定
\usepackage[top=20truemm,bottom=20truemm,left=20truemm,right=20truemm]{geometry}    %余白設定
%-----挿入設定-----%
\usepackage[dvipdfmx]{graphicx}     %図を用いる
%\usepackage[dvipdfmx]{hyperref}     %ハイパーリンクをつける
\usepackage[dvipdfmx]{color}        %色をつける
\usepackage{here}                   %場所強制コマンド[H]を使う
%-----数式設定-----%
\usepackage{amsmath,amssymb,amsthm} %数式の文字
\usepackage{amsfonts,times}         %数式のフォント
\usepackage{mathrsfs}               %花文字を使う
\usepackage{bm}                     %太字(ベクトル)を使う
\usepackage{cases}                  %場合分けを使う
\usepackage{url}                    %URLを入力する
\usepackage{float}                  %図を用いる
\usepackage{subcaption}             %サブキャプションを使う
\usepackage{mathtools}
%-----行間設定-----%
\usepackage{setspace}               %行間設定
\setstretch{0.8}
\usepackage{fancyhdr}               %-数式前後の行間設定
\setlength{\columnseprule}{0pt}
%-----題名設定-----%
\theoremstyle{definition}
\newtheorem{definition}{定義}
\newtheorem{theorem}{定理}
\renewcommand{\figurename}{Fig. }
\renewcommand{\tablename}{TABLE}
%sectionのサイズ
\usepackage{titlesec}
\titleformat*{\section}{\large\bfseries}
\titleformat*{\subsection}{\normalsize\bfseries}
\newcommand{\beforesection}[1]{\vspace{-#1mm}}
%表のサイズ
\usepackage{scalefnt}
%algorithm
\usepackage{algorithmic}
\usepackage{algorithm}
\renewcommand{\algorithmicrequire}{\textbf{Input:}}
\renewcommand{\algorithmicensure}{\textbf{Output:}}

%------------------------------%
\begin{document}

\begin{center}
{\Large Practice of MATLAB Classes}\\
\end{center}
\begin{flushright}
\begin{tabular}{rr}
{\tiny \ } &\ \\
作成日&\today\\
{\tiny \ } &\ \\
\end{tabular}
\end{flushright}

\begin{figure}[htbp]
  \begin{algorithm}[H]
  \caption{MATLABにおけるオブジェクト指向プログラミングの手順}
  \begin{algorithmic}[1] 
  \REQUIRE MATLAB環境
  \ENSURE オブジェクト指向プログラミングの理解
  \STATE クラスの概念を理解する
  \STATE 関連する属性とメソッドを持つクラスを定義する
  \STATE 属性には位置やサイズなどの特性を設定する
  \STATE メソッドにはオブジェクトの動作を記述する
  \STATE インスタンス(クラスの具体的な例)を作成する
  \STATE インスタンスを用いてメソッドを呼び出し、オブジェクトを操作する
  \STATE 必要に応じてプライベートプロパティを用いて属性を保護する
  \STATE オブジェクトの振る舞いをテストし、正しく機能することを確認する
  \end{algorithmic}
  \end{algorithm}
\end{figure}

\begin{figure}[htbp]
  \begin{algorithm}[H]
  \caption{MATLABにおけるオブジェクト指向設計の学習プロセス}
  \begin{algorithmic}[1] % 行番号をつけるために[1]を使用
  \REQUIRE MATLABユーザーがオブジェクト指向の基本的なイメージを持っていること
  \ENSURE オブジェクト指向設計の理解とMATLABでの実装能力
  \STATE オブジェクト指向の基本的なイメージを理解する
  \STATE 継承の概念を学ぶ
  \STATE クラス設計の基本思想について学ぶ
  \STATE MATLABで継承クラス設計を理解する
  \STATE クラスのアクセス権について学ぶ(private, protected, public)
  \STATE 実際のMATLABコード例を試す(例:自由落下運動のシミュレーション)
  \STATE オブジェクト指向設計の利点を理解する
  \STATE 継承の必要性について考え、適切な場面での使用を決定する
  \STATE MATLABのクラス機能についてさらに学ぶ(必要に応じて)
  \end{algorithmic}
  \end{algorithm}
\end{figure}


\end{document}