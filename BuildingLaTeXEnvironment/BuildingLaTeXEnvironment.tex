\documentclass[a4]{jarticle}  %文字サイズ・用紙サイズ・段組設定
\usepackage[top=20truemm,bottom=20truemm,left=20truemm,right=20truemm]{geometry}    %余白設定
%-----挿入設定-----%
\usepackage[dvipdfmx]{graphicx}     %図を用いる
%\usepackage[dvipdfmx]{hyperref}     %ハイパーリンクをつける
\usepackage[dvipdfmx]{color}        %色をつける
\usepackage{here}                   %場所強制コマンド[H]を使う
%-----数式設定-----%
\usepackage{amsmath,amssymb,amsthm} %数式の文字
\usepackage{amsfonts,times}         %数式のフォント
\usepackage{mathrsfs}               %花文字を使う
\usepackage{bm}                     %太字(ベクトル)を使う
\usepackage{cases}                  %場合分けを使う
\usepackage{url}                    %URLを入力する
\usepackage{float}                  %図を用いる
\usepackage{subcaption}             %サブキャプションを使う
\usepackage{mathtools}
%-----行間設定-----%
\usepackage{setspace}               %行間設定
\setstretch{0.8}
\usepackage{fancyhdr}               %-数式前後の行間設定
\setlength{\columnseprule}{0pt}
%-----題名設定-----%
\theoremstyle{definition}
\newtheorem{definition}{定義}
\newtheorem{theorem}{定理}
\renewcommand{\figurename}{Fig. }
\renewcommand{\tablename}{TABLE}
%sectionのサイズ
\usepackage{titlesec}
\titleformat*{\section}{\large\bfseries}
\titleformat*{\subsection}{\normalsize\bfseries}
\newcommand{\beforesection}[1]{\vspace{-#1mm}}
%表のサイズ
\usepackage{scalefnt}
%algorithm
\usepackage{algorithmic}
\usepackage{algorithm}
\renewcommand{\algorithmicrequire}{\textbf{Input:}}
\renewcommand{\algorithmicensure}{\textbf{Output:}}

%------------------------------%
\begin{document}

\begin{center}
{\Large Building LaTeX Environment}\\
\end{center}
\begin{flushright}
\begin{tabular}{rr}
{\tiny \ } &\ \\
作成日&\today\\
{\tiny \ } &\ \\
\end{tabular}
\end{flushright}

\begin{figure}[htbp]
  \begin{algorithm}[H]
  \caption{VSCodeでのLaTeX環境構築手順}
  \begin{algorithmic}[1]
  \REQUIRE Visual Studio Code, インターネット接続
  \ENSURE LaTeX環境が構築される
  \STATE LaTeXのインストール:
  \begin{itemize}
    \item Windows: 「install-tl-windows.exe」をダウンロードしてインストール
    \item Mac: Terminalで「brew install mactex-no-gui --cask」実行
    \item Linux (Ubuntu): Terminalで「sudo apt install texlive-full」実行
  \end{itemize}
  \STATE Visual Studio Codeのインストール:
  \begin{itemize}
    \item Windows/Mac/Linuxに応じて公式サイトからインストール
  \end{itemize}
  \STATE latexmkの設定ファイル(.latexmkrc)をユーザーホームディレクトリに作成・編集
  \STATE LaTeX Workshop拡張機能のインストール:
  \begin{itemize}
    \item VSCodeの拡張機能パネルから「LaTeX Workshop」を検索してインストール
  \end{itemize}
  \STATE VSCodeの設定ファイル(settings.json)の編集
  \STATE ユーザースニペットの登録と編集
  \STATE LaTeXソースの執筆とビルド:
  \begin{itemize}
    \item 新しいTeXファイルを作成し、スニペットを使用して執筆
    \item ビルドとPDFプレビューを行う
  \end{itemize}
  \STATE PDFからTeXソースへのジャンプ機能の使用
  \end{algorithmic}
  \end{algorithm}
\end{figure}


\end{document}