\documentclass[10pt,a4paper]{jarticle}  %文字サイズ・用紙サイズ・段組設定
\usepackage[top=20truemm,bottom=20truemm,left=20truemm,right=20truemm]{geometry}    %余白設定
%-----挿入設定-----%
\usepackage[dvipdfmx]{graphicx}     %図を用いる
\usepackage[dvipdfmx]{hyperref}     %ハイパーリンクをつける
\usepackage[dvipdfmx]{color}        %色をつける
\usepackage{here}                   %場所強制コマンド[H]を使う
%-----数式設定-----%
\usepackage{amsmath,amssymb,amsthm} %数式の文字
\usepackage{amsfonts,times}         %数式のフォント
\usepackage{mathrsfs}               %花文字を使う
\usepackage{bm}                     %太字(ベクトル)を使う
\usepackage{cases}                  %場合分けを使う
\usepackage{url}                    %URLを入力する
\usepackage{float}                  %図を用いる
\usepackage{subcaption}             %サブキャプションを使う
%-----行間設定-----%
\usepackage{setspace}               %行間設定
\setstretch{0.8}
\usepackage{fancyhdr}               %-数式前後の行間設定
\setlength{\columnseprule}{0pt}
%-----題名設定-----%
\theoremstyle{definition}
\newtheorem{definition}{定義}
\newtheorem{theorem}{定理}
\renewcommand{\figurename}{Fig. }
\renewcommand{\tablename}{TABLE}

%------------------------------%
\begin{document}
\title{第1回 進捗報告書}
\author{山田 太郎}
\date{2021/1/1}
\maketitle

\section{前回の課題}
\begin{itemize}
    \item 前回までの課題\cite{guo2022pid}.
\end{itemize}

\section{進捗状況}
\subsection{やったこと1}
\begin{definition}
    定義を書く.
\end{definition}


\subsection{やったこと2}
\begin{theorem}
    定理を書く.
\end{theorem}

\section{次回までの課題}
\begin{itemize}
    \item 次回までの課題.
\end{itemize}

%\bibliography{reference}       %bibtexで参考文献を入れる場合使用
%\bibliographystyle{IEEEtran}   %参考文献のスタイル指定

\end{document}